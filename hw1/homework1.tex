\documentclass[11pt]{article}
\usepackage{enumitem}
\usepackage{float}
\usepackage[margin=1in]{geometry}
\usepackage{graphicx}
\usepackage[space]{grffile}
\usepackage{adjustbox}
\usepackage{amsmath}
\usepackage{amsthm}
\usepackage{amssymb}
\usepackage{fullpage}
\usepackage{fancyhdr}
\usepackage{xparse}
\newcommand*{\boxtex}[1]{\framebox{#1}}
\newcommand{\cnum}{CM146}
\newcommand{\ced}{Fall 2018}
\newcommand{\ctitle}[3]{\title{\vspace{-0.5in}\cnum, \ced\\Problem Set #1: #2}}
\newcommand{\solution}[1]{{{\color{blue}{\bf Solution:} {#1}}}}
\NewDocumentCommand{\texcod}{mm}{%
	\texttt{\textcolor{#1}{#2}}%
}
\usepackage[usenames,dvipsnames,svgnames,table,hyperref]{xcolor}
\usepackage{listings}
\usepackage{color} %red, green, blue, yellow, cyan, magenta, black, white
\definecolor{mygreen}{RGB}{28,172,0} % color values Red, Green, Blue
\definecolor{mylilas}{RGB}{170,55,241}

\renewcommand*{\theenumi}{\alph{enumi}}
\renewcommand*\labelenumi{(\theenumi)}
\renewcommand*{\theenumii}{\roman{enumii}}
\renewcommand*\labelenumii{\theenumii.}

\author{Zheng Wang (404855295)}
\date{\today}
\title{MATH 151B Homework 1}

\begin{document}
	
\lstset{language=Matlab,%
	basicstyle=\footnotesize,
	breaklines=true,%
	morekeywords={matlab2tikz},
	keywordstyle=\color{blue},%
	morekeywords=[2]{1}, keywordstyle=[2]{\color{black}},
	identifierstyle=\color{black},%
	stringstyle=\color{mylilas},
	commentstyle=\color{black},%
	showstringspaces=false,%without this there will be a symbol in the places where there is a space
	numbers=left,%
	numberstyle={\tiny \color{black}},% size of the numbers
	numbersep=9pt, % this defines how far the numbers are from the text
	emph=[1]{for,end,break},emphstyle=[1]\color{red}, %some words to emphasise
	%emph=[2]{word1,word2}, emphstyle=[2]{style},    
}

\maketitle
\section*{Question 1}
\begin{itemize}
	\item [(a)]
	Let $ D = \{(t,y) \, | \, 1\le t\le 2, -\infty <y<\infty\} $. Obviously, $ f(t,y) = \frac{dy}{dt} = \frac{1+y}{t} $ is continous for all $ (t,y) \in D $.\\
	Moreover, since
	\[ \left| \frac{\partial f}{\partial y} (t,y) \right| = \left| \frac{\partial}{\partial y} \frac{1+y}{t} \right| = \left| \frac{1}{t} \right| \le 1,\quad \text{for all } (t,y)\in D  \]
	Thus, the $ f $ satisfy Lipschitz condition on $ D $ in the variable $ y $, and the IVP 
	\[ \frac{dy}{dt} =\frac{1+y}{t}, \quad1\le t\le 2, \quad y(1)=2\]
	is well-posed. \hfill $ \blacksquare $
	
	\item [(b)]
	Let $ D = \{(t,y) \, | \, 0\le t\le 1, -\infty <y<\infty\} $. Obviously, $ f(t,y) = \frac{dy}{dt} = y\cos(t) $ is continous for all $ (t,y) \in D $.\\
	Moreover, since
	\[ \left| \frac{\partial f}{\partial y} (t,y) \right| = \left| \cos(t) \, \frac{\partial}{\partial y} y \right| = \left| \cos(t) \right| \le 1,\quad \text{for all } (t,y)\in D  \]
	Thus, the $ f $ satisfy Lipschitz condition on $ D $ in the variable $ y $, and the IVP 
	\[ \frac{dy}{dt} =y\cos(t), \quad0\le t\le 1, \quad y(0)=1\]
	is well-posed. \hfill $ \blacksquare $
\end{itemize}

\section*{Question 2}
\begin{itemize}
	\item [(a)]
	Using the Euler's Method, we have the following
	\[
	\begin{cases}
	w_{i+1}=w_i + hf(t_i,w_i)\\
	w_0 = \alpha = 2
	\end{cases}
	\]
	Then, we have $ y(1.5) = w_1 = w_0 + hf(t_0,w_0) = 2 + 0.5 \times f(1,2) = 2 + \frac{1}{2}\times \frac{1+1}{1+2} = \frac{7}{3} $.\\
	Secondly, we have $ y(2) = w_2 = w_1 + hf(t_1,w_1) = \frac{7}{3} + \frac{1}{2}\times f \left( 1.5,\frac{7}{3} \right) = \frac{65}{24}$.\\
	Thus, we have \boxtex{$ y(2) = \frac{65}{24} \approx 2.70833 $}
	\item [(b)]
	For $ h = 0.5 $, $ y(2)\approx 2.708333$.\\
	For $ h = 0.2 $, $ y(2)\approx 2.729166$.\\
	For $ h = 0.1 $, $ y(2)\approx 2.735541$.\\
	For $ h = 0.01 $, $ y(2)\approx 2.741057 $.\\
	
	\item [(c)]
	\begin{table}[h]
		\centering
		\begin{tabular}{c|cccc}
			$ h $ & 0.5 & 0.2 & 0.1 & 0.01    \\ \hline
			$ y(2) $ & 2.708333 & 2.729166 & 2.735541 & 2.741057 \\ \hline
			Error ($ e $) & 0.033324 & 0.012491 & 0.006116 & 0.00060
		\end{tabular}
	\end{table}
	The exact result of $ y(2) = \sqrt{2^2+2 \times 2+6} -1  \approx 2.741657$. From the summary table above, we can see that when $ h $ get closer to $ 0 $, the approximation of $ y(2) $ generated by Euler's method will get closer to the exact value of $ y(2) $.
\end{itemize}

\section*{Question 3}
\begin{itemize}
	\item [(a)]
	We have the following by the chain rule:
	\[ \frac{df}{dt} = \frac{\partial f}{\partial t}\frac{dt}{dt} + \frac{\partial f}{\partial y}\frac{dy}{dt} = -y^2e^{-t} + (2ye^{-t})(y^2e^{-t}) = -y^2e^{-t} + 2y^3e^{-2t}\]
	
	\item [(b)]
	Euler's method give us the following formula
	\[
	\begin{cases}
	w_{i+1}=w_i + hf(t_i,w_i)\\
	w_0 = \alpha = 1
	\end{cases}
	\]
	Then, we have $ y(0.5) = w_1 = w_0 + hf(t_0,w_0) = 1 + 0.5 \times f(0,1) = \frac{3}{2} $.\\
	Secondly, we have $ y(1) = w_2 = w_1 + hf(t_1,w_1) = \frac{3}{2} + 0.5\times f \left(0.5 ,\frac{3}{2} \right) \approx 2.18235$.\\
	Thus, we have approximation from Euler's method: \boxtex{$ y(1) \approx 2.18235 $}.\\\\
	Taylor method of order 2 give us the following formula
	\[
	\begin{cases}
	w_{i+1} = w_i + hw_i^2e^{-t_i} + \frac{h^2}{2} (-w_i^2e^{-t_i} + 2w_i^3e^{-2t_i})\\
	w_0 = \alpha = 1
	\end{cases} 
	 \]
	 Then, we have $ y(0.5) = w_1 = 1 + 0.5\times1^2e^0 + \frac{0.5^2}{2}(-1^2e^0 + 2\times1^3e^0) = \frac{13}{8} $.\\
	 Secondly, we have\\ $ y(1) = w_2 = \frac{13}{8} + 0.5\times\left(\frac{13}{8}\right)^2e^{-0.5} + \frac{0.5^2}{2}(-\left(\frac{13}{8}\right)^2e^{-0.5} + 2\times \left(\frac{13}{8}\right)^3 e^{-2\times 0.5}) \approx 2.62025 $.\\
	 Thus, we have approximation from Taylor's method of order 2: \boxtex{$ y(1)\approx2.62025 $}.
	 
	 \item [(c)]
	 When $ h = 0.5 $, $ y(1) $ approximated by Euler's method is \boxtex{$ 2.182347 $}, $ y(1) $ approximated by Taylor method of order 2 is \boxtex{$ 2.620252 $}.\\
	 When $ h = 0.1 $, $ y(1) $ approximated by Euler's method is \boxtex{$ 2.531887 $}, $ y(1) $ approximated by Taylor method of order 2 is \boxtex{$ 2.711460 $}.\\
	 When $ h = 0.01 $, $ y(1) $ approximated by Euler's method is \boxtex{$ 2.695519 $}, $ y(1) $ approximated by Taylor method of order 2 is \boxtex{$ 2.718205 $}.\\\\
	 
	 \textbf{The output is generated by the following code:}
	 \lstinputlisting{Euler.m}
	 
	 \item [(d)]
	 The exact solution of $ y(1) = e^1 = e \approx 2.718282 $.\\
	 The result from Euler's method is summarized in the following table:
	 \begin{table}[h]
	 	\centering
	 	\begin{tabular}{c|ccc}
	 		$ h $ & 0.5 & 0.1 & 0.01    \\ \hline
	 		$ y(1) $ & 2.182347 & 2.531887 & 2.695519 \\ \hline
	 		Error ($ e $) & 0.535935 & 0.186395 & 0.022763
	 	\end{tabular}
	 \end{table}\\
 	The result from Taylor method of order 2 is summarized in the following table:
 	\begin{table}[h]
 		\centering
 		\begin{tabular}{c|ccc}
 			$ h $ & 0.5 & 0.1 & 0.01    \\ \hline
 			$ y(1) $ & 2.620252 & 2.711460 & 2.718205 \\ \hline
 			Error ($ e $) & 0.098030 & 0.006822 & 0.000077
 		\end{tabular}
 	\end{table}\\
	 We can see that the while the error of the estimation decrease for both method as $ h $ becomes smaller. The error of Taylor method of order 2 is much smaller than the error of Euler's Method. Moreover, the error of Taylor method of order 2 decrease much faster as $ h $ decrease than the error of Euler's method.
\end{itemize}

\end{document}
